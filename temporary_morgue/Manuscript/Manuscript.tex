\documentclass[]{elsarticle} %review=doublespace preprint=single 5p=2 column
%%% Begin My package additions %%%%%%%%%%%%%%%%%%%
\usepackage[hyphens]{url}

  \journal{An awesome journal} % Sets Journal name


\usepackage{lineno} % add
\providecommand{\tightlist}{%
  \setlength{\itemsep}{0pt}\setlength{\parskip}{0pt}}

\bibliographystyle{elsarticle-harv}
\biboptions{sort&compress} % For natbib
\usepackage{graphicx}
\usepackage{booktabs} % book-quality tables
%%%%%%%%%%%%%%%% end my additions to header

\usepackage[T1]{fontenc}
\usepackage{lmodern}
\usepackage{amssymb,amsmath}
\usepackage{ifxetex,ifluatex}
\usepackage{fixltx2e} % provides \textsubscript
% use upquote if available, for straight quotes in verbatim environments
\IfFileExists{upquote.sty}{\usepackage{upquote}}{}
\ifnum 0\ifxetex 1\fi\ifluatex 1\fi=0 % if pdftex
  \usepackage[utf8]{inputenc}
\else % if luatex or xelatex
  \usepackage{fontspec}
  \ifxetex
    \usepackage{xltxtra,xunicode}
  \fi
  \defaultfontfeatures{Mapping=tex-text,Scale=MatchLowercase}
  \newcommand{\euro}{€}
\fi
% use microtype if available
\IfFileExists{microtype.sty}{\usepackage{microtype}}{}
\ifxetex
  \usepackage[setpagesize=false, % page size defined by xetex
              unicode=false, % unicode breaks when used with xetex
              xetex]{hyperref}
\else
  \usepackage[unicode=true]{hyperref}
\fi
\hypersetup{breaklinks=true,
            bookmarks=true,
            pdfauthor={},
            pdftitle={Phenotypic evolution is less constrained in complex food webs},
            colorlinks=true,
            urlcolor=blue,
            linkcolor=magenta,
            pdfborder={0 0 0}}
\urlstyle{same}  % don't use monospace font for urls

\setcounter{secnumdepth}{0}
% Pandoc toggle for numbering sections (defaults to be off)
\setcounter{secnumdepth}{0}
% Pandoc header



\begin{document}
\begin{frontmatter}

  \title{Phenotypic evolution is less constrained in complex food webs}
    \author[a,b]{Matthew A. Barbour\corref{c1}}
   \ead{matthew.barbour@ieu.uzh.ch} 
   \cortext[c1]{Corresponding Author}
    \author[a,c]{Christopher J. Greyson-Gaito}
  
  
    \author[a]{Arezoo Sootodeh}
  
  
    \author[d]{Brendan Locke}
  
  
    \author[b]{Jordi Bascompte}
  
  
      \address[a]{University of British Columbia, Department of Zoology, 6270 University
Blvd., Vancouver, BC, V6T 1Z4, Canada}
    \address[b]{University of Zurich, Department of Evolutionary Biology and
Environmental Studies, Winterthurerstrasse 190, Zurich, 8057,
Switzerland}
    \address[c]{University of Guelph, Department of Integrative Biology, 50 Stone Rd.
East, Guelph, ONT, N1G 2W1, Canada}
    \address[d]{Humboldt State University, Department of Biological Sciences, 1 Harpst
St., Arcata, CA, 95521, USA}
  
  \begin{abstract}
  This is the abstract.
  
  It consists of two paragraphs.
  \end{abstract}
  
 \end{frontmatter}

\section{Introduction}\label{introduction}

The fitness landscape provides a unifying framework for linking the
ecology and evolution of populations (Lande 2007; McPeek 2017). The
average fitness of a population is a common currency in ecology and
evolution, but usually goes by different names in each field. Ecologists
refer to it as per-capita population growth rate (\(dN/Ndt\)), whereas
evolutionary biologists call it the natural log of population mean
fitness (\(ln(\bar W_N)\). In addition to having different names,
ecologists and evolutionary biologists have typically focused on
different processes that shape the fitness landscape. For example,
population ecologists have long studied the effect of a population's
density on its per-capita growth rate (i.e.~density-dependence, CITE
Foundational and current work). In contrast, evolutionary biologists
have focused on how the mean trait value of a population influences its
average fitness, as this describes the direction and magnitude of
natural selection (CITE foundational and current work). Therefore, the
fitness landscape describes the joint ecological and evolutionary
dynamics of a population in a given environment.

Community ecologists have extended the ecological side of the fitness
landscape by incorporating network theory. Species-interaction networks,
such as a food web describing who eats whom, provide an explicit
representation of the biotic environment as they describe the
interdependency of populations within an ecological community. This has
provided an effective framework for predicting how changes in the biotic
environment (e.g.~density of directly and indirectly connected species)
will impact population dynamics within species-rich communities. At the
same time, evolutionary biologists have long recognized that changes in
the biotic environment can alter the dynamics of natural selection.
However, the biotic environment in which populations are evolving often
remains a bit of a ``black box'' that's labelled by a general ecological
process such as competition, predation, or mutualism. Because of this,
it remains difficult to predict how changes in the biotic environment
will affect the direction and magnitude of natural selection. Such
predictions are urgently needed given the rapid changes in the biotic
environment that most populations are currently experiencing throughout
the world.

Here, we integrate species-interaction networks and the fitness
landscape to empirically test how changes in the biotic environment --
network of species interactions -- affect the dynamics of natural
selection. Specifically, we conducted a field experiment that
manipulated the diversity of insect parasitoids that were able to impose
selection on an abundant insect herbivore (\emph{Iteomyia
salicisverruca})(Fig. 1). The larva of this herbivore species induce
tooth-shaped galls when they feed on the developing leaves of willow
trees (\emph{Salix} sp., ({\textbf{???}})). These galls provide
protection from generalist predators (e.g.~ants, spiders), thus the
network of interacting parasitoids provides a realistic representation
of the biotic environment this insect herbivore is experiencing.
Therefore, our manipulation of parasitoid diversity alters the diversity
of interactions, or food-web complexity, that this insect herbivore
experiences.

Changes in food-web complexity could influence a resource population's
fitness landscape in at least two ways. First, if a more diverse
community of consumers is more effective at suppressing resource
densities (({\textbf{???}})), then this will result in lower mean
fitness of the resource population. A reduction in mean fitness, all
else equal, will intensify natural selection (({\textbf{???}})). On the
other hand, if consumers impose different selection pressures on
resource traits, then more diverse communities could dampen the strength
of selection acting on a given trait. This is because a greater
diversity in selection pressures is equivalent to greater uncertainty in
the selective environment. Thus, a more diverse consumer community may
relax the net selection pressures acting on resource traits. Here, we
evaluate these hypothesized relationships through an experimental test
of how changes in food-web complexity alters the fitness landscape of a
resource population.

\section{Materials \& Methods}\label{materials-methods}

\subsection{Study Site}\label{study-site}

We conducted our study within a four-year old common garden of coastal
willow (\emph{Salix hookeriana}) located at Humboldt Bay National
Wildlife Refuge (HBNWR) (40°40'53``N, 124°12'4''W) near Loleta,
California, USA. This common garden consists of 26 different willow
genotypes that were collected from a single population of willows
growing around Humboldt Bay. Stem cuttings of each genotype (25
replicates per genotypes) were planted in a completely randomized design
in two hectares of a former cattle pasture at HBNWR. Willows in our
garden begin flowering in February and reach their peak growth in early
August. During this study, willows had reached 5 - 9m in height. Further
details on the genotyping and planting of the common garden are
available in ({\textbf{???}}).

\subsection{Food-Web Manipulation}\label{food-web-manipulation}

We setup our food-web manipulation across 128 plants soon after galls
began developing on \emph{S. hookeriana} in early June of 2013. These
128 plants came from eight different plant genotypes, spanning the range
of trait variation observed in this willow population
(({\textbf{???}})). On treatment plants (8 replicates per genotype), we
enclosed 14 galled leaves with 10x15cm organza bags (ULINE, Pleasant
Prairie, WI, USA) to exclude three parasitoid species that attack during
larva development (hereafter larval parasitoids). This treatment did not
exclude the egg parasitoid \emph{Platygaster} sp. which attacks prior to
gall initiation (note that in Cecidomyiid midges, larva initiate gall
development CITE). On control plants (8 replicates per genotype), we
used flagging tape to mark 14 galled leaves per plant
(\textasciitilde{}30 larva), allowing the full suite of parasitoids to
attack \emph{Iteomyia}. Marking galls with flagging tape ensured that we
compared control and treatment galls with similar phenology when we
collected galls later in the season. Our food-web manipulation altered
the average number of trophic interactions that \emph{Iteomyia} was
exposed to from BLANK on control plants to BLANK on treatment plants.
Thus, we refer to galls on control plants as being exposed to a
`complex' food web, whereas galls on treatment plants were exposed to a
`simple' food web. In late August, we collected marked and bagged galls
from each plant, placed them into 30 mL vials and kept them in the lab
for 4 months at room temperature. We then opened galls under a
dissecting scope and determined whether larva survived to pupation (our
measure of fitness) or were parasitized. Since we were interested in
selection imposed by interactions with parasitoids, we restricted our
data to larva that either survived to pupation, was parasitized by an
egg parasitoid (\emph{Platygaster} sp.), or was parasitized by a larval
parasitoid. For the food-web treatment that excluded parasitoids, we
further restricted our data by removing any instances of parasitism by a
larval parasitoid. This represented less than 3\% of the observations in
this food-web treatment and allowed us to focus our inferences of
selection on those imposed by the egg parasitoid.\\
Together, we had survival estimates for 1,306 larva from 607 galls, 111
plants, and 8 plant genotypes.

\subsection{Measuring Gall Traits}\label{measuring-gall-traits}

We collected data on three different traits that we anticipated would
experience selection based on our previous work (({\textbf{???}})) and
others work with Cecidomyiid midges (({\textbf{???}}),
({\textbf{???}})). First, we measured gall diameter as the size of each
gall chamber to the nearest 0.01 mm at its maximum diameter
(perpendicular to the direction of plant tissue growth). Our previous
work has shown that a larger gall diameter provides a refuge for larva
from parasitoid attack (({\textbf{???}})). Second, we measured the
clutch size of adult female midges by counting the number of chambers in
each gall (({\textbf{???}})). All larva collected from the same
multi-chambered gall were scored with the same clutch size. Third, we
measured female preference for oviposition (egg-laying) sites as the
density of larva observed on a plant in an independent survey.
Specifically, we randomly sampled five branches per tree and summed the
number of individual gall chambers observed. We then converted these
counts to a measure of gall density per 100 shoots by counting the
number of shoots on the last branch we sampled. All larva collected from
the same plant were scored with the same female preference. The
measurement of larval densities on plants in the field is a commonly
used index for measuring oviposition preference (({\textbf{???}}));
however, caution must be taken in inferring `preference' as larval
densities can be influenced by processes other than preference
(({\textbf{???}})). Fortunately, a couple of features of our study
system suggest that larval density on a plant may be a good proxy for
female preference. For example, since our data comes from a randomized
placement of willow genotypes in a common garden, there is no consistent
bias in which willow genotypes that females are exposed to while
searching for oviposition sites. Also, egg predation is a minor source
of mortality for galling insects in general (({\textbf{???}})), thus we
do not expect any prior egg predation to bias our estimates of observed
larval densities.

\subsubsection{Quantifying the Fitness
Landscape}\label{quantifying-the-fitness-landscape}

To characterize the shape of the fitness landscape in simple and complex
food webs, we first used a generalized linear mixed model to quantify
selection surfaces on individual traits. We used a binomial error
distribution (logit link function) since larval survival (0 or 1) was
our response variable and measure of fitness. We specified linear and
quadratic terms for each gall trait as well as linear interaction terms
between each gall trait as fixed effects in the statistical models. To
account for the correlated structure of clutch size (gall level) and
female preference (plant level) as well as any independent effects of
willow genotype on larval survival, we specified gall ID nested within
plant ID nested within plant genotype as random effects. Since we were
interested in characterizing the fitness landscape -- the relationship
between mean trait values and population mean fitness -- we assumed the
mean value of our random effects (i.e.~setting them to zero) to estimate
selection gradients. Also, the fitness landscape assumes that traits
distributions are multivariate normal. To better meet this assumption,
we log-transformed clutch size and added a small constant (1) to female
preference before log transforming, since our surveys occassionaly
estimated zero larval densities. We then scaled all phenotypic traits to
mean=0 and SD=1 in order to calculate standardized selection gradients
that were comparable across traits and with other studies of natural
selection. We used the method of ({\textbf{???}}) to calculate
directional (\(\beta_{z_i}\)), quadratic (\(\gamma_{z_i,z_i}\)), and
correlational (\(\gamma_{z_i,z_j}\)) selection gradients and used
parametric bootstrapping (1000 replicates) to calculate their 95\%
confidence intervals (({\textbf{???}})). We estimated directional
selection gradients by excluding quadratic terms and statistical
interactions in the model. Note that for visualizing the fitness
landscape we restrict trait axes to \(\pm 1\) SD of the mean trait value
as this contains the majority of the trait distribution that selection
is acting on.

Rather than imposing selection, parasitoids may themselves influence the
expression of herbivore traits. Any influence on trait expression would
bias selection gradients acting on those traits. In our system, it was
plausible that parasitoids may influence chamber growth by promoting
larval feeding (cite), speeding up larva development (cite), or killing
larva before they complete their development (cite). Therefore, our
estimates of selection on chamber diameter may be positively or
negatively biased. To estimate this bias, we subset our data to only
include galls where there was variation in larval survival (1
\textgreater{} survival \textgreater{} 0) within the same gall. We then
calculated ``apparent'' selection differentials for each gall by
comparing the average chamber diameter of all larva (before
``selection'') to the average chamber diameter of surviving larva and
analyzed separate one-sample t-tests for each food-web treatment. This
analysis is based on the assumption that larva within each gall come
from the same clutch and therefore should have similar chamber diameters
regardless of whether they are parasitized. In general, we found that
our estimates of directional selection on chamber diameter were
positively biased (Appendix). In other words, our analyses were
overestimating the magnitude of selection acting on gall diameter.
Therefore, we adjusted our estimates of directional selection on chamber
diameter (\(\beta_{diam}\)) by subtracting the biased selection
differentials. Note that selection gradients and selection differentials
for chamber diameter were virtually the same (Appendix).

\subsubsection{Quantifying Selective
Constraints}\label{quantifying-selective-constraints}

The strength and pattern of selective constraints can be measured as the
slope and curvature of the fitness landscape (Arnold 1992).

We can translate selection surfaces of individuals to the fitness
landscape of a population

To characterize the net effects of food-web complexity on the slope and
curvature of \emph{Iteomyia}'s fitness landscape, we took advantage of
existing theory that links selection surfaces of individuals to the
fitness landscape of the population (Phillips \& Arnold 1998, Arnold
2003). Specifically, the slope of the fitness landscape corresponds to
the column vector of directional selection gradients:

Note that we ommitted the upper triangle of the matrix for clarity since
it is simply the reflection of the lower triangle. Assuming that there
is additive genetic variance and covariance between these traits under
selection, then the slope and curvature of the fitness landscape give
insight to how the population's mean trait value will change in the next
generation as well as how additive genetic variance and covariance
changes within a generation.

While making quantitative predictions about trait evolution requires
knowledge of the additive genetic variance and covariance of these
traits, the slope and curvature of the fitness landscape still give
qualitative insight to the evolutionary trajectory of a population.

If we assume that there is additive genetic variance and covariance
between these traits, then the matrix describing the curvature of the
fitness landscape gives qualitative insight to the selective constraints
acting on the population. For example, the diagonal of the curvature
matrix dictates (qualitatively) whether the additive genetic variance in
each trait will increase (\(+\)), decrease (\(-\)), or stay the same
(\(0\)). Similarly, the off-diagonal of the curvature matrix dictates
whether selection favors trait integration (positive covariance), a
tradeoff (negative covariance), or no change in genetic covariance. In
other words, we can get qualitative insight to how food-web complexity
influences constraints on the fitness landscape by counting the number
of negative sign values along the diagnoal (which imply a decrease in
additive genetic variance) and the number of positive or negative signs
along the off diagonal (which imply changes in additive genetic
covariance that lead to either trait integration or tradeoffs).

All analyses and visualizations were conducted in R (({\textbf{???}})).

I need to go back to estimate a common alpha coefficient if the
treatments do not differ from each other.

\section*{References}\label{references}
\addcontentsline{toc}{section}{References}

\end{document}


