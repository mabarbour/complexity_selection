\documentclass[]{elsarticle} %review=doublespace preprint=single 5p=2 column
%%% Begin My package additions %%%%%%%%%%%%%%%%%%%
\usepackage[hyphens]{url}
\usepackage{lineno} % add
\providecommand{\tightlist}{%
  \setlength{\itemsep}{0pt}\setlength{\parskip}{0pt}}

\bibliographystyle{elsarticle-harv}
\biboptions{sort&compress} % For natbib
\usepackage{graphicx}
\usepackage{booktabs} % book-quality tables
%% Redefines the elsarticle footer
%\makeatletter
%\def\ps@pprintTitle{%
% \let\@oddhead\@empty
% \let\@evenhead\@empty
% \def\@oddfoot{\it \hfill\today}%
% \let\@evenfoot\@oddfoot}
%\makeatother

% A modified page layout
\textwidth 6.75in
\oddsidemargin -0.15in
\evensidemargin -0.15in
\textheight 9in
\topmargin -0.5in
%%%%%%%%%%%%%%%% end my additions to header

\usepackage[T1]{fontenc}
\usepackage{lmodern}
\usepackage{amssymb,amsmath}
\usepackage{ifxetex,ifluatex}
\usepackage{fixltx2e} % provides \textsubscript
% use upquote if available, for straight quotes in verbatim environments
\IfFileExists{upquote.sty}{\usepackage{upquote}}{}
\ifnum 0\ifxetex 1\fi\ifluatex 1\fi=0 % if pdftex
  \usepackage[utf8]{inputenc}
\else % if luatex or xelatex
  \usepackage{fontspec}
  \ifxetex
    \usepackage{xltxtra,xunicode}
  \fi
  \defaultfontfeatures{Mapping=tex-text,Scale=MatchLowercase}
  \newcommand{\euro}{€}
\fi
% use microtype if available
\IfFileExists{microtype.sty}{\usepackage{microtype}}{}
\ifxetex
  \usepackage[setpagesize=false, % page size defined by xetex
              unicode=false, % unicode breaks when used with xetex
              xetex]{hyperref}
\else
  \usepackage[unicode=true]{hyperref}
\fi
\hypersetup{breaklinks=true,
            bookmarks=true,
            pdfauthor={},
            pdftitle={Food-web complexity flattens the fitness landscape of an insect herbivore},
            colorlinks=true,
            urlcolor=blue,
            linkcolor=magenta,
            pdfborder={0 0 0}}
\urlstyle{same}  % don't use monospace font for urls
\setlength{\parindent}{0pt}
\setlength{\parskip}{6pt plus 2pt minus 1pt}
\setlength{\emergencystretch}{3em}  % prevent overfull lines
\setcounter{secnumdepth}{0}
% Pandoc toggle for numbering sections (defaults to be off)
\setcounter{secnumdepth}{0}
% Pandoc header


\usepackage[nomarkers]{endfloat}

\begin{document}
\begin{frontmatter}

  \title{Food-web complexity flattens the fitness landscape of an insect
herbivore}
    \author[Department of Evolutionary Biology and Environmental Studies, University
of Zurich, Zurich, Switzerland]{Matthew A. Barbour\corref{c1}}
   \ead{matthew.barbour@ieu.uzh.ch} 
   \cortext[c1]{Corresponding Author}
    \author[Another University]{Jordi Bascompte}
   \ead{jordi.bascompte@ieu.uzh.ch} 
  
      \address[Some Institute of Technology]{Department, Street, City, State, Zip}
    \address[Another University]{Department, Street, City, State, Zip}
  
  \begin{abstract}
  true
  \end{abstract}
  
 \end{frontmatter}

\emph{Text based on elsarticle sample manuscript, see
\url{http://www.elsevier.com/author-schemas/latex-instructions\#elsarticle}}

\section{Introduction}\label{introduction}

Biological diversity -- from genes, to phenotypes, to species -- has
fascinated evolutionary biologists for decades.

Much of this biological diversity has been shaped by natural selection
via trophic interactions, such as resource competition (cite charater
displacement), mutualistic exchanges of resource (cite? ), and predation
(cite recent Heath and Stireman paper).

NEED to recognize that evolutionary biologists have begun to explore how
community context affects evolutionary change (work by Sharon Strauss,
Casey terHorst, Lutz Becks, etc.). These results have begun to show
interesting patterns whereby the composition of species in a community
can alter the direction and strength of natural selection imposed on
species embedded within these communities (cite).

Put another way, these results have begun to show biological diversity,
in terms of differences between species, can drive evolutionary change.

While there is clear evidence for pairwise trophic interactions to drive
evolution by natural selection, its unclear how biological diversity
itself imposes natural selection and drives evolutionary change.

Exploring the effects of biological diversity requires an explicit
examination of the network structure of trophic interactions between
species in a community. Theoretical models have begun to examine how the
network structure of species interactions drives evolutionary change
(Nuismer paper; Guimeras paper; Ecology Letters paper from a spanish
guy\ldots{}); however, we are currently lacking experimental tests.

In contrast, ecologists have begun to embrace the complexity of the
natural world, and seeking to identify the complex networks of
interactions that underlie community structure and ecosystem function.
However, these studies have not examined how evolutionary processes
feedback to shape the structure and evolution of these interaction
networks.

Food-web complexity may influence selection gradients in at least two
ways. First, if more diverse predator communities are more efficient as
suppressing prey densities (e.g.~biodiversity-ecosystem fucntion; Ives
2005 Ecology Letters), then this will result in lower mean fitness. A
reduction in mean fitness, all else equal, will intensify natural
selection and thus could increase the rate of evolutionary change.
Alternatively, if predators are functionally distinct, more diverse
communities can reduce the strength of selection. This is because each
predator has a different functional relationship between prey phenotype
and the probability of an interaction.

Here, we conducted a field experiment to test the effect of food-web
complexity on the fitness landscape of a species embedded within this
food web. To do this, we used a common garden experiment with a host
plant (\emph{Salix hooeriana}), an abundant herbivore (\emph{Iteomyia
salicisverruca}), and the diverse community of insect parasitoids that
attack it. Prior work in this system has shown that there is directional
selection for larger galls, likely because larger galls provide more of
a refuge from parasitoid attack. However, there is also evidence that
different parasitoid species impose differential selection on gall
phenotypes.

\section{Materials \& Methods}\label{materials-methods}

\subsection{Study Site}\label{study-site}

We conducted our study within a four-year old common garden of coastal
willow (\emph{Salix hookeriana}) located at Humboldt Bay National
Wildlife Refuge (HBNWR) (40°40'53``N, 124°12'4''W) near Loleta,
California, USA. This common garden consists of 26 different willow
genotypes that were collected from a single population of willows
growing around Humboldt Bay. Stem cuttings of each genotype (25
replicates per genotypes) were planted in a completely randomized design
in two hectares of a former cattle pasture at HBNWR. Willows in our
garden begin flowering in February and reach their peak growth in early
August. During this study, willows had reached 5 - 9m in height. Further
details on the genotyping and planting of the common garden are
available in Barbour et al. (2015).

\subsection{Food-Web Manipulation}\label{food-web-manipulation}

We setup our food-web manipulation across 128 plants soon after galls
began developing on \emph{S. hookeriana} in early June of 2013. These
128 plants came from eight different plant genotypes, spanning the range
of trait variation observed in this willow population (Barbour et al.
(2015)). On treatment plants (8 replicates per genotype), we enclosed 14
galled leaves with organza bags (MANUFACTORER DETAILS) to exclude three
parasitoid species that attack during larva development (hereafter
larval parasitoids). This treatment did not exclude the egg parasitoid
\emph{Platygaster} sp. which attacks prior to gall initiation (note that
in Cecidomyiid midges, larva initiate gall development CITE). On control
plants (8 replicates per genotype), we used flagging tape to mark 14
galled leaves per plant, allowing the full suite of parasitoids to
attack \emph{Iteomyia}. Marking galls with flagging tape ensured that we
compared control and treatment galls with similar phenology when we
collected galls later in the season. Our food-web manipulation altered
the average number of trophic interactions that \emph{Iteomyia} was
exposed to from BLANK on control plants to BLANK on treatment plants.
Thus, we refer to galls on control plants as being exposed to a
`complex' food web, whereas galls on treatment plants were exposed to a
`simple' food web. In late August, we collected marked and bagged galls
from each plant, placed them into 30 mL vials and kept them in the lab
for 4 months at room temperature. We then opened galls under a
dissecting scope and determined whether larva survived to pupation (our
measure of fitness) or were parasitized.

\subsection{Measuring Gall Traits}\label{measuring-gall-traits}

We collected data on three different traits that we anticipated would
experience selection based on our previous work (Barbour et al. (2016))
and others work with Cecidomyiid midges (Weis, Price, and Lynch (1983),
Heath, Abbot, and Stireman (2018)). First, we measured gall diameter as
the size of each gall chamber to the nearest 0.01 mm at its maximum
diameter (perpendicular to the direction of plant tissue growth). Our
previous work has shown that a larger gall diameter provides a refuge
for larva from parasitoid attack (Barbour et al. (2016)). Second, we
measured the clutch size of adult female midges by counting the number
of chambers in each gall (Weis, Price, and Lynch (1983)). All larva
collected from the same multi-chambered gall were scored with the same
clutch size. Third, we measured female preference for oviposition
(egg-laying) sites as the density of larva observed on a plant. The
measurement of larval densities on plants in the field is a commonly
used index for measuring oviposition preference (Gripenberg et al.
(2010)), although caution must be taken in inferring `preference'
(Singer (1986)). This is because larval densities can be influenced by
processes other than preference. For example, if an ovipositing female
is not exposed to the full spectrum of plant types (in this case
genotypes), then it is difficult to infer whether patterns of larval
densities are actually due to preference. Also, observed larval
densities could be influenced by egg predation.\\
While we recognize these limitations, a couple of aspects of our study
system likely alleviate these limitations. For example, since our data
comes from a randomized placement of willow genotypes in a common
garden, there is no consistent bias in which willow genotypes that
females are exposed to while searching for oviposition sites. Although
we cannot control for egg predation, this source of mortality appears to
play comparatively minor role in determining the mortality of galling
insects (Hawkins, Cornell, and Hochberg (1997)). To quantify female
preference (gall density), we randomly sampled five branches per tree
and summed the number of individual gall chambers observed. We converted
these counts to a measure of gall density per 100 shoots by counting the
number of shoots on the last branch we sampled. All larva collected from
the same plant were scored with the same female preference.

\subsection{Statistical Analyses}\label{statistical-analyses}

To identify the appropriate level of model complexity for testing the
effects of food-web complexity on the fitness landscapes, we compared
models using Aikaike Information Criteria. The maximal complexity we
explored was a generalized additive mixed model that fit cubic splines
to each trait as well as linear interactions between a maximum of two
traits as well as an interaction with food-web treatment. We then
examined simpler models where

This analysis enabled us to explore the appropriate complexity of the
model to include for our analyses (i.e.~include non-linear and
correlational selection gradients and whether they varied with food-web
treatment).

We used generalized additive mixed models (GAMMs, cite Bolker et al.
2008) to test the effects of food-web complexity on the shape of fitness
landscape. Larva survival (0 or 1) was our response variable and measure
of fitness. We specified our food-web treatment, each gall trait, and
all possible statistical interactions, as fixed effects to fully explore
the effects of food-web complexity on the fitness landscape. This
analysis implicitly assumes that selection is linear, which we felt was
a necessary trade-off for exploring the shape of the fitness landscape.
We specified plant genotype, plant individual nested within genotype,
and multi-chambered gall nested within plant individual, as random
effects.

To quantify selection gradients, we fit separate statistical models to
data from each food-web treatment. We used the method of Frederic J
Janzen and Hal S Stearn (1998) to calculate selection gradients and used
parametric bootstrapping to calculate their 95\% confidence intervals
(Bolker et al. (2009)).

To account for the correlated structure of our gall phenotypes
(oviposition preference at plant-level; clutch size at gall-level; gall
diameter at chamber-level), we specified gall ID nested within plant ID
nested within plant genotype as random intercepts in our statistical
models.

From these GAMMs, we estimate selection gradients by assuming the mean
value of our random effects (i.e.~setting them to zero). This was
appropriate for our analysis, since we were interested in estimating the
fitness landscape, which is function of population mean fitness and mean
trait values.

\section{Results}\label{results}

We found that more phenotypic traits were under selection in the simple
vs.~complex food web. In both complex and simple food webs, gall
diameter was under strong directional selection, with larger galls
resulting in higher larval survival (complex Beta = ; simple Beta =
)(Fig. 2A). In complex food webs, there was no evidence of selection on
clutch size (\(\beta_{clutch}=\)) or female preference
(\(\beta_{preference}=\))(orange lines in Fig. 2B,C). In simple food
webs, however, clutch size and female preference were under strong
directional selection, with smaller clutch sizes and weaker preferences
resulting in higher larval survival (blue lines in Fig. 2B,C). These
different selection pressures resulted in different adaptive landscapes
in complex vs.~simple food webs, with evidence for more rugged
landscapes in the simple rather than the complex food web (Fig. 3).
Depending on the trait combinations used to create the landscape, we
found that the ruggedness of the adaptive landscape ranged from 10\%
higher (Fig. 3A) to 274\% higher (Fig. 3C) in the simple vs.~complex
food web. Our model comparison suggested that it was unnecessary to test
for the effects of non-linear or correlational selection gradients (ref.
supp. mat.).

\section{Discussion}\label{discussion}

There are various bibliography styles available. You can select the
style of your choice in the preamble of this document. These styles are
Elsevier styles based on standard styles like Harvard and Vancouver.
Please use BibTeX~to generate your bibliography and include DOIs
whenever available.

Here are two sample references: ({\textbf{???}}; {\textbf{???}}).

\section*{References}\label{references}
\addcontentsline{toc}{section}{References}

\hypertarget{refs}{}
\hypertarget{ref-Barbour2016}{}
Barbour, Matthew A, Miguel A Fortuna, Jordi Bascompte, Joshua R
Nicholson, Riitta Julkunen-Tiitto, Erik S Jules, and Gregory M
Crutsinger. 2016. ``Genetic Specificity of a Plant--insect Food Web:
Implications for Linking Genetic Variation to Network Complexity.''
\emph{Proceedings of the National Academy of Sciences} 113 (8). National
Acad Sciences: 2128--33.

\hypertarget{ref-Barbour2015}{}
Barbour, Matthew A, Mariano A Rodriguez-Cabal, Elizabeth T Wu, Riitta
Julkunen-Tiitto, Carol E Ritland, Allyson E Miscampbell, Erik S Jules,
and Gregory M Crutsinger. 2015. ``Multiple Plant Traits Shape the
Genetic Basis of Herbivore Community Assembly.'' \emph{Funct. Ecol.} 29
(8): 995--1006.

\hypertarget{ref-Bolker2009}{}
Bolker, Benjamin M, Mollie E Brooks, Connie J Clark, Shane W Geange,
John R Poulsen, M Henry H Stevens, and Jada-Simone S White. 2009.
``Generalized Linear Mixed Models: A Practical Guide for Ecology and
Evolution.'' \emph{Trends Ecol. Evol.} 24 (3): 127--35.

\hypertarget{ref-Janzen1998}{}
Frederic J Janzen and Hal S Stearn. 1998. ``Logistic Regression for
Empirical Studies of Multivariate Selection.'' \emph{Evolution} 52 (6):
1564--71.

\hypertarget{ref-Gripenberg2010}{}
Gripenberg, Sofia, Peter J Mayhew, Mark Parnell, and Tomas Roslin. 2010.
``A Meta-Analysis of Preference-Performance Relationships in
Phytophagous Insects.'' \emph{Ecol. Lett.} 13 (3): 383--93.

\hypertarget{ref-Hawkins1997}{}
Hawkins, Bradford A, Howard V Cornell, and Michael E Hochberg. 1997.
``Predators, Parasitoids, and Pathogens as Mortality Agents in
Phytophagous Insect Populations.'' \emph{Ecology} 78 (7). Ecological
Society of America: 2145--52.

\hypertarget{ref-Heath2018}{}
Heath, Jeremy J, Patrick Abbot, and John O Stireman 3rd. 2018.
``Adaptive Divergence in a Defense Symbiosis Driven from the Top down.''
\emph{Am. Nat.} 192 (1): E21--E36.

\hypertarget{ref-Singer1986}{}
Singer, Michael C. 1986. ``The Definition and Measurement of Oviposition
Preference in Plant-Feeding Insects.'' In \emph{Insect-Plant
Interactions}, edited by James R Miller and Thomas A Miller, 65--94. New
York, NY: Springer New York.

\hypertarget{ref-Weis1983}{}
Weis, Arthur E, Peter W Price, and Michael Lynch. 1983. ``Selective
Pressures on Clutch Size in the Gall Maker Asteromyia Carbonifera.''
\emph{Ecology} 64 (4). Ecological Society of America: 688--95.

\end{document}


