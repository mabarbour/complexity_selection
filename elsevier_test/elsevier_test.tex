\documentclass[]{elsarticle} %review=doublespace preprint=single 5p=2 column
%%% Begin My package additions %%%%%%%%%%%%%%%%%%%
\usepackage[hyphens]{url}
\usepackage{lineno} % add
\providecommand{\tightlist}{%
  \setlength{\itemsep}{0pt}\setlength{\parskip}{0pt}}

\bibliographystyle{elsarticle-harv}
\biboptions{sort&compress} % For natbib
\usepackage{graphicx}
\usepackage{booktabs} % book-quality tables
%% Redefines the elsarticle footer
%\makeatletter
%\def\ps@pprintTitle{%
% \let\@oddhead\@empty
% \let\@evenhead\@empty
% \def\@oddfoot{\it \hfill\today}%
% \let\@evenfoot\@oddfoot}
%\makeatother

% A modified page layout
\textwidth 6.75in
\oddsidemargin -0.15in
\evensidemargin -0.15in
\textheight 9in
\topmargin -0.5in
%%%%%%%%%%%%%%%% end my additions to header

\usepackage[T1]{fontenc}
\usepackage{lmodern}
\usepackage{amssymb,amsmath}
\usepackage{ifxetex,ifluatex}
\usepackage{fixltx2e} % provides \textsubscript
% use upquote if available, for straight quotes in verbatim environments
\IfFileExists{upquote.sty}{\usepackage{upquote}}{}
\ifnum 0\ifxetex 1\fi\ifluatex 1\fi=0 % if pdftex
  \usepackage[utf8]{inputenc}
\else % if luatex or xelatex
  \usepackage{fontspec}
  \ifxetex
    \usepackage{xltxtra,xunicode}
  \fi
  \defaultfontfeatures{Mapping=tex-text,Scale=MatchLowercase}
  \newcommand{\euro}{€}
\fi
% use microtype if available
\IfFileExists{microtype.sty}{\usepackage{microtype}}{}
\ifxetex
  \usepackage[setpagesize=false, % page size defined by xetex
              unicode=false, % unicode breaks when used with xetex
              xetex]{hyperref}
\else
  \usepackage[unicode=true]{hyperref}
\fi
\hypersetup{breaklinks=true,
            bookmarks=true,
            pdfauthor={},
            pdftitle={Short Paper},
            colorlinks=true,
            urlcolor=blue,
            linkcolor=magenta,
            pdfborder={0 0 0}}
\urlstyle{same}  % don't use monospace font for urls
\setlength{\parindent}{0pt}
\setlength{\parskip}{6pt plus 2pt minus 1pt}
\setlength{\emergencystretch}{3em}  % prevent overfull lines
\setcounter{secnumdepth}{0}
% Pandoc toggle for numbering sections (defaults to be off)
\setcounter{secnumdepth}{0}
% Pandoc header


\usepackage[nomarkers]{endfloat}

\begin{document}
\begin{frontmatter}

  \title{Short Paper}
    \author[Some Institute of Technology]{Alice Anonymous\corref{c1}}
   \ead{alice@example.com} 
   \cortext[c1]{Corresponding Author}
    \author[Another University]{Bob Security}
   \ead{bob@example.com} 
  
      \address[Some Institute of Technology]{Department, Street, City, State, Zip}
    \address[Another University]{Department, Street, City, State, Zip}
  
  \begin{abstract}
  Studies of natural selection and fitness landscapes usually treat the
  network of interacting species as a ``black box''. Given that the loss
  of biodiversity is simplifying the structure of ecological networks,
  there is a pressing need to answer the question: how does network
  complexity affect natural selection and the fitness landscape of
  associated species? To answer this question, we conducted a field
  experiment that manipulated the complexity of a food web associated with
  a galling insect herbivore. To maintain complex food webs, we allowed
  the entire community of natural enemies to attack insect galls on 64
  plants in a common garden setting. To create simple food webs, we
  excluded a guild of three larval parasitoids by bagging galls on 64
  different plants; therefore, mortality in this treatment was primarily
  due to a single egg parasitoid that attacks prior to gall formation. We
  then measured herbivore survival as a function of three key gall traits
  in each treatment. We found that more traits were under selection in the
  simple vs.~complex food web. This occurred because different parasitoid
  species impose different selection pressures on gall traits, thereby
  minimizing relative fitness differences among insect galls with
  different phenotypes. Our work suggests that more complex food webs
  allow phenotypic variation to persist, which could facilitate subsequent
  adaptive evolution to environmental change.
  \end{abstract}
  
 \end{frontmatter}

\emph{Text based on elsarticle sample manuscript, see
\url{http://www.elsevier.com/author-schemas/latex-instructions\#elsarticle}}

\section{Introduction}\label{introduction}

Biological diversity -- from genes, to phenotypes, to species -- has
fascinated evolutionary biologists for decades. However, the explicit
context (i.e.~ecological environment) is often abstracted and not
examined in detail in studies of natural selection. Thus, we have a poor
understanding of how biological diversity itself, imposes natural
selection, and the potential feedbacks that may emerge.

In contrast, ecologists have begun to embrace the complexity of the
natural world, and seeking to identify the complex networks of
interactions that underlie community structure and ecosystem function.
However, these studies have not examined how evolutionary processes
feedback to shape the structure and evolution of these interaction
networks.

We sought to bridge this gap through a field experiment that examines
how food-web complexity alters the fitness landscape of species embedded
within this food web.

\section{Materials \& Methods}\label{materials-methods}

To isolate the effects of coastal willow (\textit{S. hookeriana} Barratt
ex Hooker) genetic variation on the plant-insect food web, we used a
common garden experiment consisting of 26 different willow genotypes (13
males; 13 females), located at Humboldt Bay National Wildlife Refuge
(HBNWR) (40°'53``N, 124°12'4''W) near Loleta, California, USA. Willow
genotypes were collected from a single population of willows growing
around Humboldt Bay. While relatedness among these genotypes is unknown,
their phenotypes in multivariate trait space are quite distinct from
each other ({\textbf{???}}), suggesting that we can treat them as
independent from one another. This common garden was planted in February
2009 with 25 clonal replicates (i.e.~stem cuttings) of each willow
genotype in a completely randomized design in two hectares of a former
cattle pasture at HBNWR. Willows in our garden begin flowering in
February and reach their peak growth in early August. During this study,
willows had reached 2 - 4 m in height. Further details on the genotyping
and planting of the common garden are available in (Barbour et al.
2015).

We conducted our study within a common garden consisting of 26 different
willow genotypes (13 males; 13 females), located at Humboldt Bay
National Wildlife Refuge (HBNWR) (40°40'53``N, 124°12'4''W) near Loleta,
California, USA. Willow genotypes were collected from a single
population of willows growing around Humboldt Bay. While relatedness
among these genotypes is unknown, their phenotypes in multivariate trait
space are quite distinct from each other (Barbour et al. 2016),
suggesting that we can treat them as independent from one another. This
common garden was planted in February 2009 with 25 clonal replicates
(i.e.~stem cuttings) of each willow genotype in a completely randomized
design in two hectares of a former cattle pasture at HBNWR. Willows in
our garden begin flowering in February and reach their peak growth in
early August. During this study, willows had reached 2 - 4 m in height.
Further details on the genotyping and planting of the common garden are
available in (Barbour et al. 2015). We conducted this experiment across
8 different plant genotypes that span the range of trait variation
(Barbour et al. 2015).

We setup our food-web manipulation soon after galls began developing on
\textit{S. hookeriana} in early June of 2013. On treatment plants, we
enclosed 14 galled leaves with organza bags (MANUFACTORER DETAILS) to
exclude three parasitoid species that attack during larva development
(hereafter larval parasitoids). This treatment did not exclude the egg
parasitoid \textit{Platygaster} sp. which attacks prior to gall
initiation (note that in Cecidomyiid midges, larva initiate gall
development CITE). On control plants, we used flagging tape to mark 14
galled leaves per plant, allowing the full suite of parasitoids to
attack \textit{Iteomyia}. Marking galls with flagging tape ensured that
we compared control and treatment galls with similar phenology when we
collected galls later in the season. In late August 2013, we collected
marked and bagged galls from each plant. We placed galls in 30 mL
transport vials and allowed them to complete development for 4 months at
room temperature in the lab. We opened galls under a dissecting scope
and determined whether larva survived to pupation, our measure of
fitness, or were parasitized.

We collected data on three different phenotypes for each larva that have
been shown to be important in our work (cite Barbour et al. 2016) and in
other work with Cecidomyiid midges (CITE). First, we measured the size
of each gall chamber to the nearest 0.01 mm at its maximum diameter
(perpendicular to the direction of plant tissue growth). Second, we
counted the number of chambers in each gall, which is indicative of the
number of larva per gall. All larva collected from the same gall were
scored with the same `number of larva per gall' phenotype. This trait is
indicative of the clutch size of adult females. \%This trait likely
reflects a combination of both larva feeding behavior as well as adult
female preferences and clutch size (CITE Art Weis' work). Third, we
estimated gall density as the number of larva per 100 shoots. We did
this by counting the number of gall chambers on five randomly sampled
branches per tree. To account for potential differences in the number of
shoots per branch for each plant genotype (CITE other willow work), we
then counted the number of shoots on the fifth branch to estimate the
number of larva per 100 shoots for each plant. All larva collected from
the same plant were scored with the same gall density phenotype. This
phenotype is indicative of the preference of female midges for
particular plant traits (hereafter `female preference').

We used generalized additive mixed models (GAMMs, cite Bolker et al.
2008) to test the effects of food-web complexity on the shape of fitness
landscape. Larva survival (0 or 1) was our response variable and measure
of fitness. We specified our food-web treatment, each gall trait, and
all possible statistical interactions, as fixed effects to fully explore
the effects of food-web complexity on the fitness landscape. This
analysis implicitly assumes that selection is linear, which we felt was
a necessary trade-off for exploring the shape of the fitness landscape.
We specified plant genotype, plant individual nested within genotype,
and multi-chambered gall nested within plant individual, as random
effects.

To account for the correlated structure of our gall phenotypes (female
preference at plant-level; clutch size at gall-level; chamber size at
chamber-level), we specified gall ID nested within plant ID nested
within plant genotype as random intercepts in our statistical models.

From these GAMMs, we estimate selection gradients by assuming the mean
value of our random effects (i.e.~setting them to zero). This was
appropriate for our analysis, since we were interested in estimating the
fitness landscape, which is function of population mean fitness and mean
trait values.

We used a spline-based semiparametric regression (Schluter 1988,
Morrissey and Sakedra 2014). These analyses are desirable because they
lead to inferences of the form of selection that make few
\textit{a priori} assumptions (Schluter 1988), and can now be used to
quantify standardized selection gradients (Morrissey and Sakrera 2014).

We then calculated selection gradients as the partial derivatives in
absolute fitness (larva survival) with response to multivariate
phenotype using the \textit{gsg} package in R (cite Morrissey and R
project). We standardized phenotypes (mean=0, SD=1), so that first-order
derivatives correspond to the intensity of directional selection
(\(\beta\)\textsubscript{trait}), while second-order derivatives
correspond to intensity of nonlinear (\(\gamma\)\textsubscript{trait})
and correlational selection
(\(\gamma\)\textsubscript{trait$_i$,}\textsubscript{trait$_j$}) and are
thus comparable within this study as well as to others.

Our GLMMs are useful for testing the effects of food-web complexity on
the fitness landscape within the context of our experimental design;
however, coefficients from GLMMs cannot be easily converted into
quantitative estimates of selection gradients (cite Morrissey's work).
Therefore, for each treatment, we fit separate generalized additive
models (GAMs) for gall traits that we identified as being under
selection from our GLMMs. Estimating selection gradients from GAMs can
give insight to both linear and non-linear selection gradients, which we
assumed were all linear for our GLMMs, given the complexity of already
fitting up to 4-way interactions in these models. For the number of
larva per gall and gall density, we aggregated larva survival at the
gall or plant level, respectively, to avoid pseudoreplication in our
GAMs.

There are various bibliography styles available. You can select the
style of your choice in the preamble of this document. These styles are
Elsevier styles based on standard styles like Harvard and Vancouver.
Please use BibTeX~to generate your bibliography and include DOIs
whenever available.

Here are two sample references: ({\textbf{???}}; {\textbf{???}}).

\section*{References}\label{references}
\addcontentsline{toc}{section}{References}

\end{document}


